\chapter{Introduction} \label{chap:intro}

The study of galaxies, their formation, and evolution has been a cornerstone of astrophysics for many decades. One of the fundamental ways to describe galaxy properties and to infer galaxy evolution is through its structure (morphology). This concept has a long history of development, starting with the earliest observations of galaxies and continuing up to the present as one of the primary methods we use to study galaxies. This thesis, titled "Investigating Galaxy Morphology in Large Surveys Using Machine Learning", investigates:-

\begin{itemize}
    \item The development and application of novel machine learning frameworks to study galaxy morphology
    \item Application of these new tools to large publicly available surveys
    \item Using these unprecedented sample sizes to shed new light on the formation and evolution of galaxies. 
\end{itemize}

In this chapter, we will start with a brief historical outline of how galaxy morphology has been used to study galaxy formation and evolution. We will discuss the different techniques that have been used to study galaxy morphology and how they have evolved over time. Finally, we will end with a discussion of the critical drawbacks of existing techniques and gaps in existing literature, that we hope to address with this thesis. 

\section{A Brief History of Galaxy Morphology} \label{sec_intro:history}

\subsection{Early Observations} \label{sec_intro:early_obs}

The systematic study of galaxy morphology has its roots in the early $20^{th}$ century. Prior to this period, galaxies were often referred to as ``nebulae", a term used to describe any diffuse astronomical object, including galaxies, star clusters, and gas clouds. The true nature of these ``nebulae" as separate galaxies outside our own Milky Way was not widely accepted until the 1920s.


Early descriptions of galaxy structures can be traced back to the pre-telescopic era. The Persian astronomer Abd al-Rahman al-Sufi, for instance, described the Andromeda nebula as a ``small cloud" in the $10^{th}$ century \citep{kepple_98}. During this time-period, the study of galaxies remained largely descriptive, focusing on cataloging and general descriptions of structure. Prominent figures in this early period included Charles Messier, William and John Herschel, who identified and located galaxies by their resolved structure, as seen by eye.

\subsection{Key Breakthroughs Powered by Technological Advances} \label{sec_intro:technology}

Just like this thesis will aim to demonstrate, major technological advancements have always been a key component in being able to better characterize galaxy morphology. The late $19^{th}$ century and early $20^{th}$ century saw significant technological advances in many fields, including better telescopes and the advent of photography ---  allowing astronomers to examine the morphology and structure of external galaxies in more detail for the first tile\citep[e.g.,][]{wolf_08,lundmark_26}. This ultimately led to Edwin Hubble's classification system for galaxies based on their visual appearance, now known as the Hubble sequence \citep{hubble_1926}. 

The basic Hubble sequence categorized galaxies into two main types: ellipticals, and spirals.  Elliptical galaxies, characterized by their smooth, featureless light distribution and ellipsoidal shape, were further classified based on their ellipticity. Spiral galaxies, characterized by their flat, rotating disk component with spiral arms, were divided into normal spirals and barred spirals, depending on whether or not they exhibited a central bar structure. Hubble, and the astronomers who followed him, could classify most nearby bright galaxies using this system. As the $20^{th}$ century progressed, the development of morphological classification methods continued. Some revisions and refinements to the Hubble sequence were proposed, including the introduction of criteria like bars, rings, and other internal features that were prominent on photographic plates of galaxies \citep[e.g.,][]{devac_59}. Also, systems were developed to classify galaxies based on the form of their spiral arms and the clumpiness of light in these arms \citep[e.g.,][]{vbg_60,vbg_76,elmgreen_87}.

The next breakthrough in studying galaxy morphology came with the advent of photometric photometry and charged-coupled devices (CCDs) --- these further revolutionized the study of galaxy structure by allowing for detailed quantitative measurements of light distributions in galaxies. These enabled \citeauthor{de_vac_48} to realize that most massive elliptical galaxies follow the same fundamental light distribution --- now popularly referred to as the ``de Vaucouleurs profile" \citep{de_vac_48}. 

This early work by \citeauthor{de_vac_48} was later expanded on by many in the 70s and 80s. \citet{sersic_63} demonstrated that one could use a more general form of light distribution that could be used to describe both disks and massive epllipticals. \citet{kormendy_77} later used this general formalism to decompose light from a single galaxy into disk and bulge profiles. 

This large body of early breakthroughs provided the basis for modern studies of galaxy morphology. They have enabled a decades-long endeavour in measuring the light profiles of galaxies in both the nearby and distant universe, that continues even to this day. This has powered a series of breakthroughs in understanding how galaxies form and evolve, which we discuss in the next section. 

\subsection{Early Impact on Understanding Galaxy Evolution} \label{sec_intro:gal_evo}
In the mid-1900s, as astronomers began cataloging the structure of more and more galaxies, they started to notice and investigate how the morphology of galaxies was correlated with other fundamental properties of galaxies. These observed correlations powered early efforts to understand how the observed features of galaxies can be explained by invoking physics. Studying such correlations have remained important even to these day, as we will show later in Chapters \ref{ch:gamornet} and \ref{chap:morph_den}.

One of the key insights derived from such early studies was the correlation between the morphology of a galaxy and its stellar population properties. Astronomers such as \citeauthor{holmberg_58} found that nearby elliptical galaxies are typically massive, consist mainly of old stars, and exhibit little to no ongoing star formation \citep{holmberg_58}. In contrast, nearby spiral galaxies contain significant populations of young stars and are actively forming new stars. This segregation of morphology in the local Universe  suggests that the morphological type of a galaxy is closely linked to its star formation history and provides an important clue for understanding the physics of galaxy formation.

This correlation between morphology and stellar population has been used to investigate the physical processes driving galaxy evolution. For instance, it has been suggested that the cessation of star formation in early-type galaxies could be due to a lack of available gas, possibly as a result of gas consumption by star formation, gas expulsion by stellar or active galactic nuclei (AGN) feedback, or gas stripping by environmental effects. On the other hand, the ongoing star formation in late-type galaxies suggests that these galaxies have been able to retain or accrete gas, possibly due to their lower masses, lower densities, or more isolated environments. We refer the interested reader to \citet{morph_review} for a recent review of these topics. We will also explore this topic a bit more in Chapter \ref{ch:gamornet}.

Another early insight that was observed happened to be the relationship between galaxy's morphology and its local environment. This empirical relationship was first noticed by \citet{dressler_84} -- he found that elliptical galaxies are more common in high-density regions of the universe, such as the centers of galaxy clusters, while spiral galaxies are more common in low-density regions, like the outskirts of clusters or in the field. This observation suggests a strong link between a galaxy's environment and its evolution, a concept that has remained focus of research in the field of galaxy evolution. In fact, in Chapter \ref{chap:morph_den}, we will present a novel study between the size and morphology of galaxies and its correlation with the large-scale structure. 

The morphology-density relation has led to the development of several theories of galaxy evolution. For instance, it has been proposed that galaxies in high-density environments undergo interactions and mergers more frequently, leading to the disruption of spiral structures and the formation of elliptical galaxies. Alternatively, the hot, dense intracluster medium in these environments could strip gas from galaxies, quenching star formation and leading to the passive evolution of spirals into lenticulars and ellipticals.

\subsection{The Revolutionary Impact of Large Surveys \& Powerful Instruments} \label{sec_intro:large_surveys}

Following the early developments noted in the previous section, the advent of large galaxy surveys --- such as the Sloan Digital Sky Survey \citep[SDSS; ][]{sdss_tech_summary} and the Cosmic Assembly Near-infrared Deep Extragalactic Legacy Survey \citep[CANDELS; ][]{candels_1}) --- revolutionized the study of galaxy morphology. These have been powered by both increasingly larger ground-based telescopes, and powerful space-based observatories, such as the Hubble Space Telescope (HST) and JWST. In this thesis, we will use data from SDSS, CANDELS, as well as the Hyper Suprime Cam Subaru Strategic Program \citep[HSC-SSP; ][]{hsc_design}. 

The large volume of data from these surveys have enabled detailed analysis of galaxy morphology. They have powered a renaissance in the analysis of galaxy structure, allowing us to use galaxy structure as a tool for deciphering how galaxy assembly occurs over cosmic time. They have helped establish that galaxies undergo substantial evolution over time. This evolution is evident in the rapid development of the stellar mass density of galaxies at $z > 1$. Remarkably, by $z = 1$, approximately half of all stellar mass has been formed \citep[e.g.,][]{bundy_05,mortlock_11}. In addition, we observe a broad array of star-formation histories across individual galaxies, coupled with the integrated star-formation rate density in the Universe's history reaching its zenith at $z \sim 2.5$ and experiencing a decline at both higher and lower redshifts \citep[e.g.,][]{shapely_11, madau_dickinson_14}. Yet, it is not trivial to develop  a clear understanding of the forces driving the genesis and development of galaxies based on these observations. 

These large surveys have been used to expand on and refine the early results noted in \S \ref{sec_intro:gal_evo}. Galaxy morphology has been demonstrated to be connected to various additional fundamental properties of galaxies and their environment, including galaxy mass, stellar kinematics, merger history, cosmic environment, and the influence of supermassive black holes \citep[e.g.,][]{Tremaine2002TheCorrelation, pozzetti_10, wuyts_11, Huertas-Company2016MassCANDELS,powell_17, shimakawa_2021, Dimauro2022CoincidenceGrowth}. Distributions of morphological quantities alone have been used to place powerful constraints on possible galaxy formation scenarios. And when combined with other physical quantities, they have been shown to provide key insights into evolutionary processes at play or even reveal the role of new physical  mechanisms that impact evolution \citep[e.g.,][]{Kauffmann2004TheGalaxies,Weinmann2006PropertiesMass,Schawinski2007TheGalaxies,vanderWel2008TheMass,Schawinski2014TheGalaxies}.

Theoretical frameworks propose several possibilities to comprehend how galaxies form and evolve, and these are now also being explored through massive hydrodynamical simulations. Current consensus posits that the genesis of galaxies could occur through various processes. These encompass the embedding of gas in dark-matter halos, in-situ star formation within collapsed galaxies, major and minor galactic mergers, and gas accretion from the intergalactic medium. All these processes significantly impact the structure of a galaxy and thus galaxy structure and morphology represent one of the most efficient tools to study galaxy formation and evolution. In Chapters \ref{ch:gamornet} and \ref{chap:morph_den} we will delve into the impact of these processes in more detail. 

We are, now, in fact, able resolve galaxies back to $z \gtrsim 9$ using JWST \citep[e.g.,][]{labbe_23,finkelstein_23, kartaltepe_23}. It has solidified the understanding that galaxy structure is significantly different in the early Universe compared with what we see in our local universe. It also reveals that there is a progression from galaxies at the highest redshifts—which are small, peculiar, and undergoing high star-formation rates—to the relatively quiescent galaxies that we find in the nearby Universe. There have also been some surprises --- JWST has revealed an unexpected population of red galaxies that appear to have redshifts of $z \sim 7 − 9$ and are very small ($\sim 200 pc$) with high masses of $\log M/M_{\odot} > 10$.

In summary, the study of galaxy morphology has come a long way since the early days of the Hubble sequence. From the initial classification of galaxies into ellipticals, spirals, and irregulars, we have developed a nuanced understanding of galaxy morphology and its role in galaxy evolution. The advent of large galaxy surveys have opened up new avenues for research and provided key insights into the correlation of galaxy morphology with other fundamental properties of galaxies and also how galaxy morphology evolves over time. Thus, the study of galaxy morphology continues to be a vibrant field with significant impact on our understanding of galaxy formation and evolution. 


\section{Methods to Determine Galaxy Morphology \& Structural Parameters} \label{sec_intro:determining_morph}

The measurement of galaxy morphology has evolved significantly over the years, with methods ranging from visual classification to sophisticated computational techniques. This evolution has been driven by the increasing volume and complexity of astronomical data, as well as advancements in technology and computational power.

\subsection{Visual Classification \& Citizen Science Projects} \label{sec_intro:trad_morph}

The most traditional approach to classify galaxies has been using visual classification. Major systems of visual classification in use today are evolved versions of the early classification systems described previously in \S \ref{sec_intro:technology}. We refer the interested reader to \citet{buta_13} for a review of visual galaxy classification. 

Visual morphological classifications have been conducted on nearly all deep HST imaging since its inception \citep[e.g.,][]{vdb_96, lee_13, kartaltepe_15} and has been applied on JWST imaging as well \citep[e.g.,][]{kartaltepe_23}. It should be noted that there are limitations on how visual classifications can be used at higher redshifts, as it has not been conclusively established how the a galaxy's apparent morphology changes with redshift effects versus actual evolution \citep{morph_review}. Furthermore, distant galaxies resembling ``elliptical" or ``disky" don't necessarily share the same characteristics as their local counterparts. Key features like size, light profiles, color, and star-formation rates differ within the same galaxy morphological type over time. Thus, a galaxy's visual morphological type, when used by itself, does not necessarily imply a particular local galaxy type, nor does it dictate a specific formation history or scale.

In response to the ever-increasing volume of data in astronomy, the early 2010s saw the advent of citizen science projects like Galaxy Zoo \citep{gzoo_original} --- these projects used online platforms to collate votes from thousands of non-scientists to classify large numbers of galaxies. Although citizen science projects have been successful in processing many surveys over the last decade, these will fail to keep up with the upcoming data glut in astronomy with the advent of even larger surveys such as The Vera Rubin Observatory Legacy Survey of Space and Time \citep[LSST;][]{lsst}, the Nancy Grace Roman Space Telescope \citep[NGRST;][]{ngrst}, and Euclid \citep{euclid}. Moreover, reliable  classifications using citizen-science projects require a decent signal-to-noise ratio, take time to set up and execute, and require an extremely careful de-biasing of the vote shares obtained \citep[e.g.,][]{gzoo_original,gzoo_candels}. 

Summarizing, while visual classification has the advantage of being intuitive and flexible, it is also subjective and time-consuming. The classification can vary between different observers, and it is impractical for upcoming large data sets containing billion galaxy samples. Despite these limitations, visual classification has played a crucial role in the study of galaxy morphology and has provided the foundation for more advanced classification methods. It will continue to provide key insights when applied appropriately on small interesting sub-samples selected from upcoming large surveys. 

\subsection{Parametric Measurements of Galaxy Structure} \label{sec_intro:parametric_measures}

As referred to in \S \ref{sec_intro:technology}, \citeauthor{de_vac_48} and \citeauthor{sersic_63} were among the first to use integrated light profiles in order to describe galaxy structures. These involve measuring the average intensity of a galaxy at a a specific radius, and then determining how this intensity changes as a function of radius. With the advent of CCDs and increased computational power, these parametric measures were widely adopted in extragalactic astronomy. In particular, the surface brightness profile introduced by \citet{sersic_63} found wide usage. This is often referred to as the \sersic{} profile and the surface brightness for a galaxy with the profile is given by

\begin{equation}
\label{eq_intro:sersic_fn}
\Sigma(r) = \Sigma_e \exp \left[ -\kappa \left( \left( \frac{r}{R_e}\right)^{1/n} - 1 \right) \right] ,
\end{equation}

where $\Sigma_e$ is the pixel surface brightness at the effective radius $r_e$, $n$ is the \sersic{} index, which controls the concentration of the light profile, and $\kappa$ is a parameter coupled to $n$ that ensures that half of the total flux is enclosed within $R_e$. Note that $R_e$ is often referred to as the ``effective radius" or ``half-light radius" of a galaxy. The de Vaucouleurs profile is typically represented by $n = 4$ in standard canonical benchmarks, while exponential disks are characterized by $n = 1$. 

A quantitative description of galaxy morphology is typically expressed in terms of structural parameters -- all of which can be derived from Equation \ref{eq_intro:sersic_fn}. These include the brightness (integration of Eq. \ref{eq_intro:sersic_fn}), shape ($n$), and size ($R_e$) --- they serve as fundamental parameters for galaxies and play a significant role in their structural analysis. 

Although the above framework has been used extensively to study the structure of galaxies, moving beyond single-\sersic{} component determinations by using separate components to analyze galaxy sub-structure (e.g., disk, bulge, bar, etc.) can provide us additional insights into the formation mechanisms of these components: bulges, disks, and bars may be formed as a result of secular evolution \citep[e.g.,][]{kormendy_2004, genzel_2008, sellwood_2014} or due to the interaction of disk instabilities with smooth and clumpy cold streams \citep[e.g.,][]{dekel_09a,dekel_09b}. As noted earlier, the procedure of combining multiple \sersic{} profiles to describe galaxy morphology was initially introduced in the context of decomposing the light in a galaxy into disk and bulge-components \citep{kormendy_1979}. A relevant parameter for the description of bulge-disk decomposition is $L_B/L_T$, which quantifies the fraction of the total light in the galaxy that is contained within the bulge. Because the bulge-disk decomposition framework provides a better way to describe most (local) galaxies as opposed to single \sersic{} profiles, we will be using it extensively later in this work to both simulate light profiles of galaxies as well as to determine the structural parameters of real galaxies.

The above-described procedure of fitting analytic two dimensional light profiles to galaxy images found wide adoption in the early $21^{st}$ century \citep[e.g.,][]{graham_03, kormendy_09, simard_11,vdw_12}. The procedure provided a uniform framework to describe galaxy structure and effectively compare the morphology of different populations of galaxies. 

The task of fitting the above-mentioned light profiles to galaxy data is typically performed using a handful of light-profile fitting programs --- such as GIM2D \citep{gim2d}, GALFIT \citep{galfit}, GALAPAGOS \citep{galapagos}, Morfometryka \citep{morfometryka}, and ProFit \citep{profit}. There are slight differences between how these codes work; but the broad structure is the same. They depend on the user specifying a model and specifying the initial values for the parameters of the chosen light profile model (such as those described in Eq. \ref{eq_intro:sersic_fn}). The codes thereafter iteratively adjust the  parameter values of the model to minimize the residuals and improve the fit. This process is often performed using optimization algorithms, such as the Levenberg-Marquardt algorithm or a Markov Chain Monte Carlo (MCMC) method. As the optimization process continues, the code refines the parameter estimates until it converges to the best-fitting values. The convergence criteria can vary depending on the code implementation and user-defined settings.

Some of the above mentioned also predict crude estimates of uncertainties --- however as we will demonstrate in Chapter \ref{chap:hsc_morph}, these estimates are typically sever underestimations of the true uncertainty. These codes also typically suffer from the fact that the quality of the fit depends heavily on the input parameters, and when dealing with millions of galaxies, such hand-refinement of input parameters is an impossible task. We will delve deeper into some of these challenges and potential solutions upcoming sections. 

Despite these challenges, these codes represent simple methods for measuring the light profile of galaxies and have played a key role in the last two decades in helping us to understand the evolution of galaxy structure.

\subsection{Non-Parametric Measurements of Galaxy Structure} \label{sec_intro:non_parametric_measures}

Although not a focus of this thesis, for completeness, we provide a quick description of non-parametric methods to describe galaxy morphology. Non-parametric techniques broadly refer to a collection of statistical techniques that do not rely on specific predefined models or assumptions about the functional form of the galaxy's light profile. Instead, these methods aim to describe the structure of galaxies based on their observed properties.

These methods  originated in the photographic era, with early attempts by \citet{morgan_62} to quantify light concentration in galaxies. However, comprehensive quantitative measurements did not occur until the mid-1990s \citep{rix_95,conselice_97}. Currently, the most commonly used non-parametric methods for measuring galaxy structure include the concentration (C), asymmetry (A), clumpiness (S) system (CAS system), as well as similar parameters introduced by \citet{takamiya_99, papovich_03, abraham_03, lotz_04}. These parameters aim to capture the main characteristics of galaxy structures without assuming a specific underlying form, as is the case with model fitting described in \S \ref{sec_intro:parametric_measures}. Concentration measures the compactness of a galaxy's light distribution by quantifying the ratio of light contained within a specific radius to the total light. Asymmetry measures the deviation of a galaxy's appearance from perfect symmetry by comparing the pixel values of the galaxy with those of its mirror image.

Two other popular non-parametric measures happen to be the Gini Coefficient and M20 --- these have typically been used to find galaxies of broad morphological types, especially galaxies undergoing mergers \citep{abraham_03, lotz_04}. These parameters measure the relative distribution of light within pixels and do not involve subtraction, as is used for the asymmetry and clumpiness parameters, and therefore in principle may be less sensitive to high levels of background noise. The Gini coefficient assesses the inequality in the distribution of pixel values within a galaxy image. It measures the cumulative distribution function of the sorted pixel values and indicates how uniformly or unevenly the light is distributed. M20 is a measure of the concentration of the brightest $20\%$ of a galaxy's light. It quantifies the contribution of the brightest regions to the overall luminosity and helps identify galaxies with prominent central concentrations.

Note that just like parametric methods, non-parametric measures of galaxy morphology also suffer from many challenges:- a) Degeneracy and Interpretation: Non-parametric measures may not uniquely determine the underlying physical processes or structural components responsible for the observed morphology. Different galaxy structures can yield similar non-parametric measurements, leading to potential degeneracy and challenges in interpretation; b) Redshift Effects: Measuring galaxy structure at high redshifts introduces challenges due to limited resolution and sensitivity of observations. The interpretation of non-parametric measures becomes more complex as the morphological features may appear differently or be affected by observational biases; c) Subjectivity:  Non-parametric measures often involve subjective judgments or choices made during the analysis process. Decisions such as selecting regions of interest or setting thresholds can introduce variability and potential bias into the results.

\subsection{The Advent of Machine Learning} \label{sec_intro:ml_morph}

\begin{figure}[htbp]
    \centering
    \includegraphics[width=0.6\textwidth]{data_volume.png}
    \caption{This figure has been adapted from \citet{kremer_17}. It shows the rate of data that will be produced (every night) by upcoming surveys such as the Rubin Observatory Legacy Survey of Space and Time (LSST) and the Thirty Meter Telescope (TMT) in comparison to existing surveys such as the Sloan Digital Sky Survey (SDSS). Note that the y-scale is logarithmic.}
    \label{fig_intro:data_volume}
\end{figure}


Over the next decade we will witness a glut of new data in astronomy from a variety of new surveys such as Rubin-LSST, NGRST, and Euclid. Although we often talk about ``big-data" in astronomy -- it is helpful to quantify the volume of data we will be generating. Figure \ref{fig_intro:data_volume}, adapted from \citet{kremer_17}, shows that these newer surveys will be producing roughly 100 times more data every night compared to surveys in the last decade such as SDSS. Therefore, in order to keep up with the data-rate, existing algorithms would have be scaled by a factor of 100. It is safe to say that none of the methods described previously in \S \ref{sec_intro:trad_morph} - \ref{sec_intro:non_parametric_measures} can be scaled to these levels. 

Note that it is not simply a matter of computational resources that prevent the above methods from being sped up. Arguably, even if the astronomy community were to able to access 100 times more computational resources than it has in the past decade, still it would be nearly impossible to use these techniques to analyze these large data volumes. The fact of the matter is that none of the above techniques were built to handle such large data volumes --- they require manual initiation, careful tuning, visual inspection of residuals and careful post-processing for individual galaxies. 

Because of these challenges, over the last decade, machine learning has been increasingly used by astronomers to determine galaxy morphology. Early efforts in applying machine learning to galaxy morphology classification on a large scale were motivated by the data available from SDSS \citep[e.g.,][]{Ball2004GalaxyNetworks,Kelly2004MorphologicalSurvey,banerji_10}. These early methods involved the user selecting proxies, such as color, concentration index, and spectral features, as inputs to the machine learning models. However, since the relationship between these proxies and galaxy morphology was often unknown and potentially biased, these early networks were not optimal replacements for the traditional classification methods described previously. 

In the early years of the last decade, deep  convolutional neural networks\,(CNNs) revolutionized the field of image processing\,\citep[see][for an overview]{dl_1}. They are ideal for galaxy morphology classification as they eliminate the need for manual selection of morphological proxies. Instead, the network autonomously determines the most discriminative image features for distinguishing between different classes. The initial significant effort to employ CNNs for morphological classification of galaxies emerged from the "Galaxy Challenge" organized by Galaxy Zoo. Participating teams competed to replicate the vote distributions of each question in Galaxy Zoo 2 using a CNN. The top-performing entry was presented by \citealp{Dieleman2015Rotation-invariantPrediction}). This was followed by the work of \citet{Huertas-Company2015ALEARNING}, who used a CNN to reproduce visual classifications for CANDELS galaxies.

From these early attempts at using a CNN to classify galaxies morphologically (e.g.,  \citealp{Dieleman2015Rotation-invariantPrediction}) to the largest CNN produced morphology catalogs currently available \citep{Cheng2021GalaxyNetworks, Vega-Ferrero2021PushingSurvey}, most CNNs have provided broad, qualitative classifications, rather than numerical estimates of morphological parameters. Such studies typically entail classifying galaxies based on their morphological properties (e.g., based on whether the galaxy has a disk or a bulge or a bar, etc.) as opposed to predicting values of relevant morphological parameters that help characterize the galaxy (such as bulge-to-total light ratio, radius, etc.). By contrast, \citet{Tuccillo2018DeepFitting} used a CNN to estimate the parameters of a single-component \sersic{} fit, though  without uncertainties. Meanwhile, the computation of full Bayesian posteriors for different morphological parameters is crucial for drawing scientific inferences that account for uncertainty and thus are indispensable in the derivation of robust scaling relations  \citep[e.g.,][]{Bernardi2013TheProfile, vanderWel20143D-HST+CANDELS:3} or tests of theoretical models using morphology \citep[e.g.,][]{Schawinski2014TheGalaxies}. Thus, producing posterior estimates will significantly increase the scientific potential of morphological catalogs produced using CNNs. We will discuss this further in the next section. 

Using deep learning for determining galaxy morphology presents several challenges. One significant challenge is the availability and quality of labeled training data. Creating large and diverse datasets with accurate morphological annotations can be time-consuming and require expert knowledge. Another challenge is the interpretability of deep learning models. Understanding the decision-making process of complex neural networks and extracting meaningful insights from their learned representations can be difficult. Therefore, it is imperative to perform rigorous testing on such frameworks before mass-adoption. Additionally, prediction of uncertainties is known to be a significant challenge for these models. We will outline some of these challenges in more detail in the next section, followed by a description in the later chapters of how we address them.

\section{Outstanding Challenges \& Opportunities} \label{sec_intro:outstanding_challenges}
Despite the significant advancements in the study of galaxy morphology as outlined in \S \ref{sec_intro:history} \& \ref{sec_intro:determining_morph}, there remain several challenges and limitations with the current tools and methods.
These challenges are particularly pronounced in the context of large astronomical surveys. However, these challenges also provide the perfect opportunity for innovation; and when applied on an appropriate dataset can provide new insights into galaxy formation and evolution. In this section, we will outline the various challenges that we hope the address in this thesis. 

\subsection{Limitations of Existing Methods} \label{sec_intro:limitations}

Draw heavily from your papers and your job talk.

\subsection{Potential New Insights into Galaxy Evolution} \label{sec_intro:opportunities}


\begin{figure}[htbp]
    \centering
    \includegraphics[width=\textwidth]{ml_workflow.png}
    \caption{The machine learning frameworks developed in this thesis are complementary to exiting tools and methods. \textit{(Left):} adfasd fds fasdf .\textit{(Right):} adfsdf asdf }
    \label{fig_intro:ml_workflow}
\end{figure}




\section{Outline of This Thesis} \label{sec_intro:outstanding_challenges}